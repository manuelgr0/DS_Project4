% This is based on "sig-alternate.tex" V1.9 April 2009
% This file should be compiled with V2.4 of "sig-alternate.cls" April 2009
%
\documentclass{sig-alternate}

\usepackage[english]{babel}
\usepackage{graphicx}
\usepackage{tabularx}
\usepackage{subfigure}
\usepackage{enumitem}
\usepackage{url}


\usepackage{color}
\definecolor{orange}{rgb}{1,0.5,0}
\definecolor{lightgray}{rgb}{.9,.9,.9}
\definecolor{java_keyword}{rgb}{0.37, 0.08, 0.25}
\definecolor{java_string}{rgb}{0.06, 0.10, 0.98}
\definecolor{java_comment}{rgb}{0.12, 0.38, 0.18}
\definecolor{java_doc}{rgb}{0.25,0.35,0.75}

% code listings

\usepackage{listings}
\lstloadlanguages{Java}
\lstset{
	language=Java,
	basicstyle=\scriptsize\ttfamily,
	backgroundcolor=\color{lightgray},
	keywordstyle=\color{java_keyword}\bfseries,
	stringstyle=\color{java_string},
	commentstyle=\color{java_comment},
	morecomment=[s][\color{java_doc}]{/**}{*/},
	tabsize=2,
	showtabs=false,
	extendedchars=true,
	showstringspaces=false,
	showspaces=false,
	breaklines=true,
	numbers=left,
	numberstyle=\tiny,
	numbersep=6pt,
	xleftmargin=3pt,
	xrightmargin=3pt,
	framexleftmargin=3pt,
	framexrightmargin=3pt,
	captionpos=b
}

% Disable single lines at the start of a paragraph (Schusterjungen)

\clubpenalty = 10000

% Disable single lines at the end of a paragraph (Hurenkinder)

\widowpenalty = 10000
\displaywidowpenalty = 10000
 
% allows for colored, easy-to-find todos

\newcommand{\todo}[1]{\textsf{\textbf{\textcolor{orange}{[[#1]]}}}}

% consistent references: use these instead of \label and \ref

\newcommand{\lsec}[1]{\label{sec:#1}}
\newcommand{\lssec}[1]{\label{ssec:#1}}
\newcommand{\lfig}[1]{\label{fig:#1}}
\newcommand{\ltab}[1]{\label{tab:#1}}
\newcommand{\rsec}[1]{Section~\ref{sec:#1}}
\newcommand{\rssec}[1]{Section~\ref{ssec:#1}}
\newcommand{\rfig}[1]{Figure~\ref{fig:#1}}
\newcommand{\rtab}[1]{Table~\ref{tab:#1}}
\newcommand{\rlst}[1]{Listing~\ref{#1}}

\title{WiFi Direct Message Flooding API \\
\normalsize{Distributed Systems -- Project Proposal}}
% \subtitle{subtitle}

% Use the \alignauthor commands to handle the names
% and affiliations for an 'aesthetic maximum' of six authors.

\numberofauthors{1} %  in this sample file, there are a *total*
% of EIGHT authors. SIX appear on the 'first-page' (for formatting
% reasons) and the remaining two appear in the \additionalauthors section.
%
\author{
% You can go ahead and credit any number of authors here,
% e.g. one 'row of three' or two rows (consisting of one row of three
% and a second row of one, two or three).
%
% The command \alignauthor (no curly braces needed) should
% precede each author name, affiliation/snail-mail address and
% e-mail address. Additionally, tag each line of
% affiliation/address with \affaddr, and tag the
% e-mail address with \email.
%
% 1st. author
\alignauthor \normalsize{Student One,  Student Two, Student Three}\\
	\affaddr{\normalsize{ETH ID-1 XX-XXX-XXX, ETH ID-2 XX-XXX-XXX, ETH ID-3 XX-XXX-XXX}}\\
	\email{\normalsize{one@student.ethz.ch, two@student.ethz.ch, three@student.ethz.ch}}
}

\begin{document}
	
	\maketitle
	
	\begin{abstract}
		Again, its this time of the year, when all the large festivals and parades are. Its this time, when all of us give in on one of these and go there, like every year. 
You go there with a few friends and soon enough you will loose one of them, because the crowd is just too big and too loud. Every one of us knows this situation, right? \\
It's this situation when you really need to use your mobile phone, but ending up be annoyed about the absence of ANY reception. \\
So that's the point where we started... \\

We want a messaging system to work, even if you can't reach youre friends over the internet. We also wanted not only to program a messaging app for that purpose, rather then go a step further and build an API to provide these functionalities to users with even different approches than a "normal" messaging app. \\

Our approach to build such a network of nodes in an usable range for mobile devices, is to use WiFi Direct. The API will not only forward the messages to a server, it broadcasts the messages to all nodes. This gives us the highest possibility to get a message to a node without any reception. \\

So far, so good. But it's not finished yet. It's a lot more stuff needed then just broadcast a message through a network of devices. The API should allow a dynamic network structure, which means at any time a node can leave or join the network, it also should allow a kind of buffering the messages to allow reaching nodes which are not reachable for the network at that moment. This leads to a lot of challenging problems with replacement orderings, timeouts and so forth... \\

---------------------------------------------------------------------------\\
TODO: Write here if i forgot something.... !! \\

----------------------------------------------------------------------------
	\end{abstract}
	\section{Introduction}
		Our message flooding API can be useful to many future projects that involve several Android devices which should be connected even without a working internet connection. For some applications, the API might simply provide an alternative communication channel that can be used when the device does not have a connection to the internet, but for other applications it can be the core of the communication between several devices. \\
A simple example application will be distributed along with the API as a demo. The demo is an SOS forwarding app that uses our API to propagate an emergency call between devices which are not connected with the internet, until it reaches a device with a working internet connection that can send the call to a webserver. \\
Of course the full power of the API will only be visible in more complex systems. In principle, the API will be powerful enough to support a document editor which is synchronized over many users, all without the need of a working internet connection. That could be interesting for a military office outside, but also for a working team that wants to keep working on the same files while traveling together in an airplane.\\
To demonstrate how the API is used for more complex applications, we will develop a messenger app. The app will support multiple secure chats that users can join. \\
As the name suggest, the API provides nothing but a message flooding interface, therefore most of the complexity will be in the client's code outside of the API, namely in the client's application. However, the API solves most of the problems of a distributed systems and hides them from the client. The features available in the API are:
\begin{itemize}
	\item {\bf Dynamic local network}: Devices can form a local network and new devices can enter it dynamically.
	\item {\bf Message flooding}: A device can easily send a message to all other devices in the local network.
	\item {\bf Message buffering}: A device which loses connection to the other devices will receive all sent messages when it connects to the local network again.
	\item {\bf Message reordering}: The ordering of messages sent by one device is preserved on the receiver side.
\end{itemize}
There are already applications and services which, to some degree,  do the same. One such example is the FireChat app. FireChat is a proprietary mobile app that builds on a wireless decentralized mesh network to enable smartphones to connect and send messages to each other. The main difference is that we develop an API instead of one single application. While our API can be used to implement something similar to FireChat (as we will show with our Chat App example) it is capable of handling many more and very different use cases like the aforementioned SOS app.

		
		% TODO: Describe similar projects like firechat and serval project
		
	\section{System Overview}
		% This is the core of the proposal.
	% It is where you spell out your technical plan and explain the project design.
	% Expected evaluation/demonstration issues would also be addressed in this section.
	% Use helpful figures such as~\rfig{example} and~\rfig{system-overview},
	% explain the figures in the text where you reference them. 
		\subsection{API}
		Jakob, Manuel
			
			Last Contact Table:
			\begin{center}
				\begin{tabular}{ | l | l |}
					\hline
					$N_{1}$ & $T_{1}$ \\ \hline
					$N_{2}$ & $T_{2}$ \\ \hline
					\vdots & \vdots \\ \hline
					$N_{n}$ & $T_{n}$ \\ 
					\hline
				\end{tabular}
			\end{center}
			
			ACK-Table:
			\begin{center}
				\begin{tabular}{ l | l | l | l | l | l |}
					\multicolumn{5}{c}{Receiver}\\
					\cline{2-6}
					$\multirow{5}*{\rotatebox{90}{Sender}}$ & $\cellcolor[gray]{0.65}$ & $N_{1}$ & $N_{2}$ & $\hdots$ & $N_{n}$ \\ \cline{2-6}
					& $N_{1}$ & $\cellcolor[gray]{0.65}$ &  &  $\hdots$ &  \\ \cline{2-6}
					& $N_{2}$ &  & $\cellcolor[gray]{0.65}$ & $\hdots$ &  \\ \cline{2-6}
					& $\vdots$ &  &  & $\ddots$ &   \\ \cline{2-6}
					& $N_{n}$ & & & $\hdots$ & $\cellcolor[gray]{0.65}$ \\ \cline{2-6}
				\end{tabular}
			\end{center}
			
			Message:
			\begin{center}
				\begin{tabular}{ | l |}
					\hline
					Last Contact Table \\ \hline
					ACK-Table \\ \hline
					Content \\ \hline
				\end{tabular}
			\end{center}
			
			Acknowledgement:
			\begin{center}
				\begin{tabular}{ | l |}
					\hline
					Last Contact Table \\ \hline
					ACK-Table \\ \hline
				\end{tabular}
			\end{center}
			
		
		\subsection{Emergency App}
		Claude, Alessandro
		
		\subsection{Chat App}
		Joel, Pascal
			
			The Chat App ensures end to end encrypted messages via peer-to-peer connection through the flooding API. Encrypting and Decrypting messages is done public key cryptography. The keys are generated by the user and shared by QR codes that have to be scanned from the receiver.\\
			If the receiver's network is not connected to the sender's network the messages are buffered and will be sent to the receiver later when the receiver's and the sender's network are connected. The receiver is able to get as many messages as are stored in the buffer.\\
			
			When first starting the App the user has to enter his name and generate his public and private key. After generating the key the user is able to scan public keys from other members or provide his own public key for scanning. Upon scanning a new chat is displayed in the chat-list and a reminder appears to scan the public key of the chat partner. \\
			
			For group chats symmetric keys are used. Every chat will have an administrator who generates a symmetric key and can distribute the key to other group members.\\
			
			Pressing on a chat in the chat-list opens a chat to write and read messages.\\

		% Describe system setup, components, external libraries, hardware etc.
	
	\section{Requirements}
		Joel\\

Several choices have to be made that limit the reach of our application, in order to keep the project simple enough for the given time frame. Perhaps the choice that limits us most is using the Wi-Fi Peer-to-Peer API in Android. It constrains us to devices that have at least Android 4 (API level 14) installed and that have hardware capable of Wi-Fi Direct communication.\cite{P2PAPIGuide}\\
For reading and generating QR codes we will use the ZXing project\cite{ZXing}. We will prompt user to install the ZXing Barcode Scanner app if not already available. It's installation size is in the neighbourhood of 1MB depending on the device, so the size is not a problem. The app requires API level 16, access to the Google Play Store and of course a device with a camera. However we provide an alternative, if somewhat arduous, method of copying the public keys by typing them in manually.\\
Beyond that we will use only standard Java and Android libraries so no further limitations apply to the system software.\\

ENTWEDER\\

We will develop two apps, one being the emergency app, the other the chat app. Both of those will bundle the API core component, but storage concerns should be fairly insignificant since we don't include much audiovisual content in either. Running both services at once may put some strain on system memory, however Android should handle resource allocation cleanly and if need be even terminate one of the services. They may contend for Wi-Fi resources as well, so if the users notice too much performance degradation they will need to make sure that they only run one service at a time.\\

ODER\\

We will develop three apps, one being a wrapper around the core networking service, with a simple interface to enter some configuration details. The emergency app and the chat app are going to use an API to access network functionality from the first. Each app individually will be rather lightweight so storage concerns should be fairly insignificant since we don't include much audiovisual content in any of these components. In case the users wish to use one app with one group of people and the other with another, they will have to reconfigure their network association in the wrapper app repeatedly.\\

REDEWTNE\\



	
		
		
	
	\section{Work Packages}
		
	% Breakdown the work to subtasks to meet the project requirements.
	% Define and describe these tasks.
	
		\subsection{API}
			Jakob, Manuel
		
		\subsection{Emergency App}
			Claude, Alessandro
		
		\subsection{Chat App}
			Joel, Pascal
	
	\section{Milestones}
		% The milestones section provides a work plan for carrying out the project.
% This is your schedule for getting the project done.
% Clearly state how the work packages will be distributed among the team members.
First of all the public function signatures of our API are defined and handed to the other group members that they can start with the Emergency App and the Chat App. Then the API team works at the remaining work packages and the other group members can start with their work on the emergency app and the chat app. The emergency app team will partially support the API team until the work packages 1 to 9 are met.\\[3mm]
Before the emergency app and the chat app can be tested the API has to be finished because the two apps rely on the message forwarding of the API. \\[3mm]
Schedule:
\begin{center}
\scalebox{.95}{
\begin{tabular}{|l|l|l|}
		\hline
		Date: & Subject to finish: & Responsible: \\ \hline
		20 Nov & Function overview API & Manuel, Jakob \\
		24 Nov & Emergency App UI complete & Alessandro, Claude\\
		25 Nov & Chat app up to WP3 complete & Joel, Pascal \\
		4 Dec & Chat app up to WP5 complete & Joel, Pascal \\
		4 Dec & API: Basic send/recv. (up to WP9) & Manuel, Jakob \\
		10 Dec & Observable API behavior is stable & Manuel, Jakob \\
		11 Dec & Chat app complete for testing & Joel, Pascal \\
		11 Dec & Emergency App: able to set off & \\ & and display requests & Alessandro, Claude\\
		12 Dec & Projection slides & all\\
		14 Dec & Emergency App: Webservice for & \\ & distribution of requests running & Alessandro, Claude\\
		18 Dec & All tasks complete & all\\
		\hline
	\end{tabular}
}

\end{center}
		% The milestones section provides a work plan for carrying out the project.
		% This is your schedule for getting the project done.
		% Clearly state how the work packages will be distributed among the team members. 
		Pascal
	
	\section{References}
	
	\bibliographystyle{abbrv}
	\bibliography{report}
	
\end{document}
