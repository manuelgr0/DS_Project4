% This is based on "sig-alternate.tex" V1.9 April 2009
% This file should be compiled with V2.4 of "sig-alternate.cls" April 2009
%
\documentclass{report}

\usepackage[english]{babel}
\usepackage{graphicx}
\usepackage{tabularx}
\usepackage{subfigure}
\usepackage{enumitem}
\usepackage{url}


\usepackage{color}
\definecolor{orange}{rgb}{1,0.5,0}
\definecolor{lightgray}{rgb}{.9,.9,.9}
\definecolor{java_keyword}{rgb}{0.37, 0.08, 0.25}
\definecolor{java_string}{rgb}{0.06, 0.10, 0.98}
\definecolor{java_comment}{rgb}{0.12, 0.38, 0.18}
\definecolor{java_doc}{rgb}{0.25,0.35,0.75}

% code listings

\usepackage{listings}
\lstloadlanguages{Java}
\lstset{
	language=Java,
	basicstyle=\scriptsize\ttfamily,
	backgroundcolor=\color{lightgray},
	keywordstyle=\color{java_keyword}\bfseries,
	stringstyle=\color{java_string},
	commentstyle=\color{java_comment},
	morecomment=[s][\color{java_doc}]{/**}{*/},
	tabsize=2,
	showtabs=false,
	extendedchars=true,
	showstringspaces=false,
	showspaces=false,
	breaklines=true,
	numbers=left,
	numberstyle=\tiny,
	numbersep=6pt,
	xleftmargin=3pt,
	xrightmargin=3pt,
	framexleftmargin=3pt,
	framexrightmargin=3pt,
	captionpos=b
}

% Disable single lines at the start of a paragraph (Schusterjungen)

\clubpenalty = 10000

% Disable single lines at the end of a paragraph (Hurenkinder)

\widowpenalty = 10000
\displaywidowpenalty = 10000
 
% allows for colored, easy-to-find todos

\newcommand{\todo}[1]{\textsf{\textbf{\textcolor{orange}{[[#1]]}}}}

% consistent references: use these instead of \label and \ref

\newcommand{\lsec}[1]{\label{sec:#1}}
\newcommand{\lssec}[1]{\label{ssec:#1}}
\newcommand{\lfig}[1]{\label{fig:#1}}
\newcommand{\ltab}[1]{\label{tab:#1}}
\newcommand{\rsec}[1]{Section~\ref{sec:#1}}
\newcommand{\rssec}[1]{Section~\ref{ssec:#1}}
\newcommand{\rfig}[1]{Figure~\ref{fig:#1}}
\newcommand{\rtab}[1]{Table~\ref{tab:#1}}
\newcommand{\rlst}[1]{Listing~\ref{#1}}

\title{WiFi Direct Message Flooding API \\
\normalsize{Distributed Systems -- Project Proposal}}
% \subtitle{subtitle}

% Use the \alignauthor commands to handle the names
% and affiliations for an 'aesthetic maximum' of six authors.

\numberofauthors{1} %  in this sample file, there are a *total*
% of EIGHT authors. SIX appear on the 'first-page' (for formatting
% reasons) and the remaining two appear in the \additionalauthors section.
%
\author{
% You can go ahead and credit any number of authors here,
% e.g. one 'row of three' or two rows (consisting of one row of three
% and a second row of one, two or three).
%
% The command \alignauthor (no curly braces needed) should
% precede each author name, affiliation/snail-mail address and
% e-mail address. Additionally, tag each line of
% affiliation/address with \affaddr, and tag the
% e-mail address with \email.
%
% 1st. author
\alignauthor \normalsize{Student One,  Student Two, Student Three}\\
	\affaddr{\normalsize{ETH ID-1 XX-XXX-XXX, ETH ID-2 XX-XXX-XXX, ETH ID-3 XX-XXX-XXX}}\\
	\email{\normalsize{one@student.ethz.ch, two@student.ethz.ch, three@student.ethz.ch}}
}

\begin{document}
	
	\maketitle
	
	\begin{abstract}
	\end{abstract}
	\section{Introduction}
		Our message flooding API can be useful to many future projects that involve several Android devices which should be connected even without a working internet connection. For some applications, the API might simply provide an alternetive communication channel that can be used when the device does not have a connection to the internet, but for other applications it can be the core of the communication between several devices. \\
		A simple example application will be distributed along with the API as a demo. The demo is an SOS forwarding app that uses our API to propagate an emergency call between devices which are not connected with the internet, until it reaches a device with a working internet connection that can send the call to a webserver. \\
		Of course the full power of the API will only be visible in more complex systems. In principle, the API will be powerful enough to support a document editor which is synchronized over many users, all without the need of a working internet connection. That could be interesting for a military office outside, but also for a working team that wants to keep working on the same files while travelling together in an airplane. \\
		To demonstrate how the API is used for more complex applications, we will develop a messenger app. The app will support multiple secure chats that users can join. \\
		As the name suggest, the API provides nothing but a message flooding interface, therefore most of the copmplexitiy will be in the client's code outside of the API, namely in the client's application. However, the API solves most of the problems of a ditributed systems and hides them from the client. The features available in the API are:
		\begin{itemize}
        	\item {\bf Dynamic local network}: Devices can form a local network and new devices can enter it dynamically.
        	\item {\bf Message flooding}: A device can easily send a message to all other devices in the local network.
        	\item {\bf Message buffering}: A device which loses connection to the other devices will receive all sent messages when it connects to the local network again.
        	\item {\bf Message reordering}: The ordering of messages sent by one device is preserved on the receiver side.
		\end{itemize}
		% TODO: Describe similar projects like firechat and serval project
		
	\section{System Overview}
	% This is the core of the proposal.
	% It is where you spell out your technical plan and explain the project design.
	% Expected evaluation/demonstration issues would also be addressed in this section.
	% Use helpful figures such as~\rfig{example} and~\rfig{system-overview},
	% explain the figures in the text where you reference them. 
		\subsection{API}
		Jakob, Manuel
		
		\subsection{Emergency App}
		Claude, Alessandro
		
		\subsection{Chat App}
		Joel, Pascal
			
			The Chat App ensures end to end encrypted messages via peer-to-peer connection through the flooding API. Encrypting and Decrypting messages is done public key cryptography. The keys are generated by the user and shared by QR codes that have to be scanned from the receiver.\\
			If the receiver's network is not connected to the sender's network the messages are buffered and will be sent to the receiver later when the receiver's and the sender's network are connected. The receiver is able to get as many messages as are stored in the buffer.\\
			
			When first starting the App the user has to enter his name and generate his public and private key. After generating the key the user is able to scan public keys from other members or provide his own public key for scanning. Upon scanning a new chat is displayed in the chat-list and a reminder appears to scan the public key of the chat partner. \\
			
			Pressing on a chat in the chat-list opens a chat to write and read messages.\\
			
	
	\section{Requirements}
	% Describe system setup, components, external libraries, hardware etc.
		Joel
	
	\section{Work Packages}
	% Breakdown the work to subtasks to meet the project requirements.
	% Define and describe these tasks.
	
		\subsection{API}
			Jakob, Manuel
		
		\subsection{Emergency App}
			Claude, Alessandro
		
		\subsection{Chat App}
			Joel, Pascal
	
	\section{Milestones}
		% The milestones section provides a work plan for carrying out the project.
		% This is your schedule for getting the project done.
		% Clearly state how the work packages will be distributed among the team members. 
		Pascal
	
	\section{References}
	
	\bibliographystyle{abbrv}
	\bibliography{report}
	
\end{document}
