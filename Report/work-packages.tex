% Breakdown the work to subtasks to meet the project requirements.
	% Define and describe these tasks.
	
		\subsection{API}
			Jakob, Manuel

\begin{enumerate}
\item Establish Peer-to-Peer connection with WiFi Direct
\item Build data structures for LC-Table and ACK-Table
\item Implement functions to update tables
\item Build message and parse message
\item Implement broadcast\_message
\item Implement send\_message
\item Implement receive\_message (with message listener)
\item Build data structure for local buffer
\end{enumerate}


		
		\subsection{Emergency App}
			Claude, Alessandro
			\begin{enumerate}
				\item Main Activity with "request help" button. Button is only clickable if personal information is entered and location services are turned on. On button click the user can select what kind of emergency case it is.
				\item Settings Activity which stores personal information such as Name, insurance numbers, allergies, etc.
				\item Notification Activity which shows a relayed emergency request on the users phone including walk directions to find the requester.
				\item A webserver which distributes the request to the specific emergency services in charge.
			\end{enumerate}
		
		\subsection{Chat App}
			Joel, Pascal
			\begin{enumerate}
				\item MainActivity: Clickable list of chats ordered by activity with names and unread message counter overflow menu with "Preferences", "Show Key", "Add Chat", "Go Offline"
				\item Storing chats, address book and own keys in files when service is shut down
				\item ChatActivity: Chat window, with message list left and right aligned, depending on sender, ordered descending in age
				\item Using ZXing library make two activities, one for displaying keys and one for scanning them
				\item Preferences, for sound and vibration and new key generation
				\item Service that handles message state, address book state, receiving messages including decryption, notification to be started on app start
				\item Generating public-private key pair with javax.crypto
				\item Activity for initial key generation and name entry
			\end{enumerate}
			